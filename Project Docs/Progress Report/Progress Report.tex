\documentclass[aps,prb,twocolumn,superscriptaddress]{revtex4-2} % APS journal style
\usepackage[a4paper,margin=0.8in]{geometry}
\usepackage{graphicx}
\usepackage{amsmath, amssymb}
\usepackage{hyperref}
\usepackage{float}
\usepackage{booktabs}
\usepackage{siunitx}
\usepackage{caption}
\usepackage{subcaption}
\usepackage{titlesec} 
\usepackage[utf8]{inputenc}
\usepackage{enumitem}
\raggedbottom
% Bibliography commands
\bibliographystyle{apalike} % Use APS style for revtex4-2 compatibility
\titleformat{\section}{\bfseries\large\uppercase}{\thesection.}{1em}{}
\titleformat{\subsection}{\bfseries}{\thesubsection}{1em}{}

\begin{document}

\title{Quantum Error Correction resilient against Atom Loss}
\author{Adeniyi Osinowo}
\affiliation{
 Department of Physics and Astronomy,
 University College London,
 Gower Street,
 London,
 WC1E 6BT,
 UK
} 
\date{\today}

\maketitle
\begin{abstract}
    h
\end{abstract}
\section{Introduction} 
% - Explain motivation/significance of the topic: Highlight key advantages of Neutral atom arrays over other qubit architectures
In recent years, neutral atom arrays have emerged as a promising platform for quantum computing due to their scalability, long coherence times, and the ability to manipulate individual atoms with high precision. These advantages position neutral atom arrays as strong contenders against other qubit architectures such as superconducting qubits and trapped ions. However, one of the significant challenges faced by neutral atom arrays is atom loss, which can severely impact the performance of quantum error correction (QEC) strategies.
% - Define problem statement: Introduce neutral atom architecure to explain the issue of atom loss and how QEC strategies can help mitigate the issue
Atom loss in neutral atom arrays can occur due to various factors, including collisions with background gas particles, spontaneous emission, and technical imperfections in trapping mechanisms. This loss of atoms leads to errors in quantum computations, which QEC strategies aim to mitigate. Effective QEC strategies are essential for maintaining the integrity of quantum information and ensuring reliable quantum computations in the presence of atom loss.
% - Explain key research questions and how they will be explored and applied in this literature review: conduct exploratory analysis of QEC strategies that can mitigate atom loss in neutral atom arrays
%   - Identify key factors/challenges of atom loss in neutral atom arrays on QEC performance
%   - Identify relevant error models for atom loss in neutral atom arrays
%   - Identify relevant QEC strategies that can mitigate atom loss in neutral atom arrays
%   - Identify key factors to consider when assessing QEC strategies for atom loss in neutral atom arrays
This literature review aims to conduct an exploratory analysis of QEC strategies that can effectively mitigate atom loss in neutral atom arrays. The key research questions to be addressed include:
\begin{itemize}
    \item What are the key factors and challenges associated with atom loss in neutral atom arrays, and how do they impact QEC performance?
    \item What are the relevant error models for atom loss in neutral atom arrays?
    \item What QEC strategies have been proposed to mitigate atom loss in neutral atom arrays?
    \item What are the key factors to consider when assessing the effectiveness of QEC strategies for atom loss in neutral atom arrays?
\end{itemize}
\section{Methodology}
% - An update on the progress of the literature review essay - defining the extent of the review and the sources that have been identified so far.
% 	- Discuss key factors/challenges of atom loss in neutral atom arrays on QEC performance
% 	- Discuss relevant error models for atom loss in neutral atom arrays
% 	- Discuss relevant QEC strategies that can mitigate atom loss in neutral atom arrays
% 	- Discuss key factors to consider when assessing QEC strategies for atom loss in neutral atom arrays

\section{Results \& Discussion}
% - A presentation of any preliminary results
% 	- Progress on implementation of error model
% 	- Progress on implementation of relevant QEC strategies for assessment
\section{Conclusion}
% - A discussion of the difficulties encountered so far and plan for the future
% 	- Summary of progress
% 	- Outline and explain difficulties/limitations encountered so far and questions that remain unanswered
% 	- Briefly describe a revised Timeline or changes to plan
\section*{REFERENCES}
\bibliography{reference_texts} 
\end{document}